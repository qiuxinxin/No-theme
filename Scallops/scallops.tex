\documentclass[notheorems,mathserif,table,compress]{beamer}  %dvipdfm选项是关键,否则编译统统通不过
%%------------------------常用宏包------------------------
%%注意, beamer 会默认使用下列宏包: amsthm, graphicx, hyperref, color, xcolor, 等等
\usepackage{fontspec,xunicode,xltxtra}  % for XeTeX
\usepackage{verbatim}
\usepackage{mathabx}
\usepackage{amsfonts,amssymb}
\usepackage{iplouclistings}
%%------------------------ThemeColorFont------------------------
%% Presentation Themes
% \usetheme[<options>]{<name list>}
\usetheme{Madrid}
%% Inner Themes双精度计算
% \useinnertheme[<options>]{<name>}
%% Outer Themes
% \useoutertheme[<options>]{<name>}
\useoutertheme{miniframes} 
%% Color Themes 
%\usecolortheme[<options>]{<name list>}
%% Font Themes
\usefonttheme{serif}
\setbeamertemplate{background canvas}[vertical shading][bottom=white,top=structure.fg!7] %%背景色, 上25%的蓝, 过渡到下白.
\setbeamertemplate{theorems}[numbered]
\setbeamertemplate{navigation symbols}{}   %% 去掉页面下方默认的导航条.
\usepackage{zhfontcfg}
%\setsansfont[Mapping=tex-text]{文泉驿正黑}  %% 需要fontspec宏包
     %如果装了Adobe Acrobat,可在font.conf中配置Adobe字体的路径以使用其中文字体
     %也可直接使用系统中的中文字体如SimSun,SimHei,微软雅黑 等
     %原来beamer用的字体是sans family;注意Mapping的大小写,不能写错
     %设置字体时也可以直接用字体名,以下三种方式等同:
     %\setromanfont[BoldFont={黑体}]{宋体}
     %\setromanfont[BoldFont={SimHei}]{SimSun}
     %\setromanfont[BoldFont={"[simhei.ttf]"}]{"[simsun.ttc]"}
%%------------------------MISC------------------------
\graphicspath{{figures/}}         %% 图片路径. 本文的图片都放在这个文件夹里了.
%%------------------------正文------------------------
\begin{document}
\XeTeXlinebreaklocale "zh"         % 表示用中文的断行
\XeTeXlinebreakskip = 0pt plus 1pt % 多一点调整的空间
%%----------------------------------------------------------
%% This is only inserted into the PDF information catalog. Can be left
%% out.
%%%
%% Delete this, if you do not want the table of contents to pop up at
%% the beginning of each subsection:
\AtBeginSection[]{                              % 在每个Section前都会加入的Frame
  \frame<handout:0>{
    \frametitle{Contents}\small
    \tableofcontents[current,currentsubsection]
  }
}

\AtBeginSubsection[]                            % 在每个子段落之前
{
  \frame<handout:0>                             % handout:0 表示只在手稿中出现
  {
    \frametitle{Contents}\small
    \tableofcontents[current,currentsubsection] % 显示在目录中加亮的当前章节
  }
}

%%----------------------------------------------------------
\title{Learning Summary}
\author[qiu]{主讲人~~~~~\textcolor{olive}{邱欣欣}\\
    \quad 幻灯片制作~~\textcolor{olive}{邱欣欣}}
\institute[中国海洋大学]{\small\textcolor{violet}{中国海洋大学~~信息科学与工程学院}}
\date{2013~年~11~月~21~日}
%\titlegraphic{\vspace{-6em}\includegraphics[height=7cm]{ouc}\vspace{-6em}}
\frame{ \titlepage }
%%----------------------------------------------------------
\section*{Contents}
\frame{\frametitle{Learning Summary}\tableofcontents}
%%----------------------------------------------------------
\section{Scallops Heartbeat Analysis}

%
\begin{frame}
\frametitle{HHT (Hilbert-Huang Transform)}
The main contents of the HHT includes two parts:
\begin{itemize}
\item The complicated signal can be divided into limited IMFs (Intrinsic Mode Function) by the EMD (Empirical Mode Decomposition) method.
\item Using the Hilbert transform to solve the instantaneous frequency of IMFs and getting the energy-frequency-time distribution.
\end{itemize}
The advantages of the HHT:
\begin{itemize}
\item It is designed to analyze nonlinear and non-stationary data.
\item Adaptive, uniqueness of the signal decomposition and good local properties both in time-domain and frequency-domain.
\end{itemize}
\end{frame}

%
\begin{frame}
\frametitle{EMD-Empirical Mode Decomposition}
\begin{itemize}
\item The EMD is a procedure that complicated signal can be decomposed into several IMFs, it's also called the sifting process.
Assume the signal is $s(t)$,
\begin{displaymath}
s(t)=\sum^n_{i=1}c_i(t)+r_n(t)
\end{displaymath}
$c_i(t)$ is the IMF components, and the $r_n(t)$ is the residual term.
\end{itemize}
\end{frame}

%
\begin{frame}
\frametitle{EMD-Empirical Mode Decomposition}
The existing problem of EMD:
\begin{itemize}
\item Stopping criterion : Component termination conditions and Decomposition termination conditions 
\item Curve fitting method : Cubic splines interpolation
\item Boundary conditions : Endpoint extension and Mirror symmetry
\end{itemize}
\end{frame}

%
\begin{frame}
\frametitle{Scallops Heartbeat Analysis}
\XeTeXpicfile "./2-1原始信号与去除imf1-5后频谱对比.png" xscaled 350 yscaled 350
\end{frame}
%
\begin{frame}
\frametitle{Scallops Heartbeat Analysis}

\end{frame}

\section{Literature Management and Information Analysis}

%
\begin{frame}
\frametitle{What did it tell us?}
Four information practice:
\begin{itemize}
\item Information Acquisition \newline
-Information Consciousness and Search Consciousness \newline
-Search engine, Database and RSS
\item Information Management \newline
-Endnote, WIZ and Total Commander 
\item Information Analysis \newline
-HistCite
\item Sharing, Cooperation and Innovation \newline
-Mind Mapping
\end{itemize}
\end{frame}

%
\begin{frame}
\frametitle{What did it tell us?}
Three consciousness:
\begin{itemize}
\item Utilizing the tools to increase efficiency
\item Time investment \newline
-More preparation may quicken the speed in doing work.
\item Sharing consciousness \newline
-Sharing is the best learning style \newline
-Sharing can let yourself make progress \newline
-Sharing is a happy thing
\end{itemize}
\end{frame}

%
\begin{frame}
\frametitle{What did it tell us?}
\begin{itemize}
\item Develop a new method of one's own-Determine the direction
\item Go ahead boldly-Do it now
\item Perseverance-Make a sustain effort 
\item Presence of mind-Success will come when conditions are ripe
\end{itemize}
\end{frame}


%
\begin{frame}
\frametitle{What have we learned from it?}
\begin{itemize}
\item 
\item
\item 
\end{itemize}
\end{frame}

%
\begin{frame}
\frametitle{What have we learned from it?}
About the presentation-Sharing the questions and thought with others
\begin{itemize}
\item Communicating with teamers before the report, detemine the topic and ensure the continuity.
\item Try to explain the complex things as simply as we can.
\item Talk about something beyond the knowledge itself, and discuss the questions and thought with the audiences.
\end{itemize}
\end{frame}

\section{Introduction to Computing}
%
\begin{frame}
\frametitle{Introduction to Computing} 
\begin{itemize}
\item 

\end{itemize}
\end{frame}

%
\begin{frame}
\frametitle{Introduction to Computing}
\begin{itemize}
\item 

\end{itemize}
\end{frame}

%
\begin{frame}
\frametitle{Introduction to Computing}
\begin{itemize}
\item 

\end{itemize}
\end{frame}

%
\begin{frame}
\frametitle{Introduction to Computing}
\begin{itemize}
\item 

\end{itemize}
\end{frame}

%
\begin{frame}
\frametitle{Introduction to Computing} 
\begin{itemize}
\item 

\end{itemize}
\end{frame}

\end{document}
